\documentclass[../Project.tex]{subfiles}
\begin{document}
\newpage
\section{Background}
\text{ if and only if }alse
\subsection{Group Theory}
	Character theory will require a comprehensive understanding of conjugacy classes.
	\begin{defi}[Conjugacy class \cite{2}]
		Let $G$ be a group. Recall that for elements $g_1,g_2 \in G$, we say $g_1$ is conjugate to $g_2$ if there exists some $h \in G$ such that $g_2 = hg_1h^{-1}$. The \textit{conjugacy} class of $g \in G$ is the equivalence class $g^G = \Brace{hgh^{-1} \;:\; h \in G}$.
		\begin{mitem}
		\item We call an element $h \in g^G$ a \textit{representative} of $g^G$
		\item We call two conjugacy classes $g_1^G,g_2^G$ \textit{distinct} if they are disjoint
		\end{mitem}
	\end{defi}
	
	\begin{prop}
		Given two conjugacy classes $g_1^G,g_2^G$ in $G$, either $g_1^G = g_2^G$ or $g_1^G \cap g_2^G = \varnothing$.
	\end{prop}
	\begin{proo*}
		Suppose there exists $g \in g^G_1\cap g^G_2$. Then there exists $h_1,h_2 \in G$ such that $g = h_1g_1h_1^{-1} = h_2g_2h_2^{-1}$. Then $g_1 = h_1^{-1}gh_1 = h_1^{-1}h_2g_2h_2^{-1}h_1 = (h_1^{-1}h_2)g_2(h_1^{-1}h_2)^{-1}$ and $g_1^G = g_2^G$.
	\end{proo*}

	We will examine the structure of $S_4$ and $D_6$ with representation theory in later sections, and we will see why a  complete list of their conjugacy classes is helpful.

	\begin{exam}
		By direct calculation, there are five conjugacy classes of $G = S_4$. Written in cycle notation they are
		\begin{mitem}
			\item $(1_{G})^G = \Brace{1_{G}}$,
			\item $(12)^G = \Brace{(12),(13),(14),(23),(24),(34)}$,
			\item $(123)^G = \Brace{(123),(132),(124),(142),(134),(143),(234),(243)}$,
			\item $((12)(34))^G = \Brace{(12)(34),(13)(24),(14)(23)}$,
			\item $(1234)^G = \Brace{(1234),(1342),(1423),(1243),(1432),(1324)}$.
		\end{mitem}
	\end{exam}

	\begin{exam}[Conjugacy classes of $D_6$]
		\label{examconjofD6}
		Consider the dihedral group $G = D_6 = \gen{x,y\;:\;x^3,\, y^2,\,(yx)^2}$ with six elements $\Brace{1_G,y,yx,yx^2,x,x^2}$. We can calculate our conjugacy classes as 
		$$(1_G)^G = \Brace{1_G},\quad y^G = \Brace{y,yx,yx^2},\quad x^G = \Brace{x,x^2}.$$
	\end{exam}


The content on this page will be useful in our examination of $S_4$ in later sections.


\newpage
\begin{defi}[Group action \cite{4}]
	Let $G$ be a group. Recall the \textit{group action} of $G$ on a set $X$ is a map $\textasteriskcentered \cdot \textasteriskcentered : G \times X \to X$ such that
	\begin{menum}
		\item $1 \cdot x = x$ $\text{ for all } x \in X$,
		\item $g_1 \cdot (g_2 \cdot x) = g_1g_2 \cdot x\quad\text{ for all } g_1,g_2 \in G,\,x \in X$
	\end{menum}
\end{defi}

\begin{defi}[Orbit \cite{4}]
	Given a group $G$ acting on a set $X$ the \textit{orbit} of an element $x \in X$ is the set
	$$\Orb(x) \coloneqq \Brace{y \in X\;:\; \exists g \in G \text{ with } y = g \cdot x}.$$
\end{defi}

\begin{defi}[Stabilizer \cite{4}]
	Given a group $G$ acting on a set $X$ the \textit{stabilizer} of an element $x \in X$ is the set
	$$\Stab(x) \coloneqq \Brace{g \in G\;:\; g \cdot x = x}.$$
\end{defi}

\begin{theo}[Orbit-Stabilizer theorem \cite{4}]
	Given a group $G$ acting on a finite set $X$ we have
	$$\abs{\Orb(x)} = \Brack{G : \Stab(x)},$$
	the index of $\Stab(x)$ in $G$.
\end{theo}
\begin{proo*}
TODO
\end{proo*}

Orbit-Stabilizer theorem is an incredibly useful property and can be seen in a large number of contexts in group theory. We will use it to prove a relation between conjugacy classes and centralizers which will be used frequently in our study of character tables.


\begin{defi}[Centralizer \cite{3}]
	Let $G$ be a group and $g \in G$. The \textit{centralizer} $C_G$ of $g$ is the set of elements in $G$ which commute with $g$, written
	$$C_G(g) \coloneqq \Brace{h \in G\;:\;gh = hg}.$$
\end{defi}

\begin{prop}[\cite{2}]
	\label{cardofconj}
	For a finite group $G$, and a conjugacy class $g^G \subseteq G$ we have $\abs{x^G} = \dfrac{\abs{G}}{\abs{C_G(g)}}$.
\end{prop}
\begin{proo*}
	Let $G$ be a finite group, and $g,h \in G$.
	Let $\textasteriskcentered \cdot \textasteriskcentered : G \times G \to G$ be the map
	$h \cdot g = hgh^{-1}$. This is a group action of $G$ on itself. Cleary $\Orb(g) = g^G$, and $\Stab(g) = \Brace{ h \in G \;: hg = gh} = C_G(g)$, then by Orbit-Stabilizer Theorem we have
	$$\abs{g^G} = \Brack{G : C_G(g)} = \frac{\abs{G}}{\abs{C_G(g)}}.$$
\end{proo*}
\newpage

The fundamental theorem of abelian groups is a result which will allow us to classify finite abelian groups with ease.
\begin{theo}[Fundamental theorem of finite abelian groups \cite{4}]
	\label{funtheoabelian}
	Let $G$ be a finite abelian group. Then $G$ is isomorphic to a direct product of cyclic groups $C_{n_1} \oplus \cdots \oplus C_{n_k}$.
\end{theo}
\begin{proo*}
	TODO
\end{proo*}
\newpage
\fi
\subsection{The Linear Algebra Recap and the General Linear Group}

\begin{defi}[Endomorphisms and Automorphisms]
	Given a space $V$ with an algebraic structure, an \textit{endomorphism} is defined to be a homomorphism from $V$ to itself. We denote $\End(V) \coloneqq \Hom(V,V)$ to be the set of all endomorphisms on $V$. An \textit{automorphism} is an endomorphism which is also an isomorphism, and we denote $\Aut(V) \coloneqq \Brace{\phi \in \End(V)\;|\;\phi \text{ is an isomorphism}}$ as the set of all automorphisms on $V$.\\
\end{defi}

In this project, we focus on vector space endomorphisms - linear maps from a space to itself.

\begin{defi}[General linear group \cite{1}]
	Let $V$ be a vector space. We define
	$$\GL(V) \coloneqq \Aut(V)$$
	to be the set of invertible linear endomorphisms over $V$. It is clear from the properties of compostion that $\GL(V)$ is a group under compostion. We call $\GL(V)$ the \textit{general linear group} of vector space $V$.\label{1}
\end{defi}

We will most often be dealing with finite-dimensional vector spaces, and so will prefer to view the general linear group in matrix form.\\

\textit{(NOTE TO GWYN - I did the following proposition and proof on the first day of working but reviewing it now it seems pretty excessive and straightforward so I'm strongly considering taking it out)}

\begin{prop}
	 If $V$ is an $n$-dimensional vector space over a field $F$ then there is a group isomorphism
	$$\GL(V) \cong \GL_n(F) \coloneqq \Brace{A \in \M_n(F) \;|\; A \text{ is invertible}},$$
	the group of invertible $n \times n$ matrices.
\end{prop}
\begin{proo*}
	Let $V$ be an $n$-dimensional vector space over $F$ and fix a basis $\mathcal{B} = \Brace{e_1,\dots,e_n}$. Recall that the result of a linear transformation is entirely determined by its result on basis elements (once a basis is chosen). Then for a linear endomorphism $L : V \to V$, we can write the result of $L$ on basis element $e_k$ as $L(e_k) = \alpha^{(1)}_ke_1 + \cdots + \alpha^{(n)}_ke_n$. Let $\phi : \Aut(V) \to \M_n(F)$ such that
	$$\phi(L) =
	\begin{pmatrix} 
	    \alpha^{(1)}_1 & \alpha^{(1)}_2 & \dots & \alpha^{(1)}_n \\
	    \alpha^{(2)}_1 & \alpha^{(2)}_2 & \cdots & \alpha^{(2)}_n \\
	    \vdots & \vdots& \ddots & \vdots\\
	    \alpha^{(n)}_1 & \alpha^{(n)}_2  &\dots & \alpha^{(n)}_n 
	\end{pmatrix}.$$
	Any matrix has a corresponding linear map which sends the $k$th basis vector to another vector with the basis coefficients made up of the scalars in the $k$th column. Conversly, any linear map is determined by the vectors that the basis elements are mapped to, for which there is a unique linear map with columns as the coefficients. Then $\phi$ is a bijection. Also the set of invertible matrices is a group under matrix multiplication.  \\

	Now we show that $\phi$ is a homomorphism. Given $L_1(e_k) = \alpha^{(1)}_ke_1 + \cdots + \alpha^{(n)}_ke_n$ and $L_2(e_k) = \beta^{(1)}_ke_1 + \cdots + \beta^{(n)}_ke_n$, we have
	\begin{align*}
	L_2 \circ L_1(e_k) &= L_2(\alpha^{(1)}_ke_1 + \cdots + \alpha^{(n)}_ke_n) = \alpha_k^{(1)}L_2(e_1) + \cdots + \alpha_k^{(n)}L_2(e_n)\\
	&= \alpha^{(1)}_k(\beta^{(1)}_1e_1 + \cdots + \beta^{(n)}_1e_n) + \cdots + \alpha^{(n)}_k(\beta^{(1)}_ne_1 + \cdots + \beta^{(n)}_ne_n)\\
	&= e_1(\alpha^{(1)}_k\beta^{(1)}_1 + \cdots + \alpha^{(n)}_k\beta^{(1)}_n) + \cdots e_n(\alpha^{(1)}_k\beta^{(n)}_1 + \cdots +\alpha^{(n)}_k\beta^{(n)}_n).
	\end{align*}
	Therefore
	\begin{align*}
		\phi(L_2 \circ L_1) &= 
		\begin{pmatrix} 
	    (\alpha^{(1)}_1\beta^{(1)}_1 + \cdots +\alpha^{(n)}_1\beta^{(1)}_n) & (\alpha^{(1)}_2\beta^{(1)}_1 + \cdots + \alpha^{(n)}_2\beta^{(1)}_n) & \dots & (\alpha^{(1)}_n\beta^{(1)}_1 + \cdots +\alpha^{(n)}_n\beta^{(1)}_n) \\
	    (\alpha^{(1)}_1\beta^{(2)}_1 + \cdots +\alpha^{(n)}_1\beta^{(2)}_n) & (\alpha^{(1)}_2\beta^{(2)}_1 + \cdots + \alpha^{(n)}_2\beta^{(2)}_n) & \cdots & (\alpha^{(1)}_n\beta^{(2)}_1 + \cdots +\alpha^{(n)}_n\beta^{(2)}_n) \\
	    \vdots & \vdots& \ddots & \vdots\\
	    (\alpha^{(1)}_1\beta^{(n)}_1 + \cdots +\alpha^{(n)}_1\beta^{(n)}_n) & (\alpha^{(1)}_2\beta^{(n)}_1 + \cdots + \alpha^{(n)}_2\beta^{(n)}_n) &\dots & (\alpha^{(1)}_n\beta^{(n)}_1 + \cdots +\alpha^{(n)}_n\beta^{(n)}_n) 
		\end{pmatrix}\\
		&= \begin{pmatrix} 
	    \beta^{(1)}_1 & \beta^{(1)}_2 & \dots & \beta^{(1)}_n \\
	    \beta^{(2)}_1 & \beta^{(2)}_2 & \cdots & \beta^{(2)}_n \\
	    \vdots & \vdots& \ddots & \vdots\\
	    \beta^{(n)}_1 & \beta^{(n)}_2  &\dots & \beta^{(n)}_n 
	\end{pmatrix} \times
\begin{pmatrix} 
	    \alpha^{(1)}_1 & \alpha^{(1)}_2 & \dots & \alpha^{(1)}_n \\
	    \alpha^{(2)}_1 & \alpha^{(2)}_2 & \cdots & \alpha^{(2)}_n \\
	    \vdots & \vdots& \ddots & \vdots\\
	    \alpha^{(n)}_1 & \alpha^{(n)}_2  &\dots & \alpha^{(n)}_n 
	\end{pmatrix}
		= \phi(L_2) \times \phi(L_1).
	\end{align*}
	$\blacksquare$
\end{proo*}

\text{ if and only if }alse
The following theorem is essential for a counterexample to Maschke's theorem in later chapters.
\begin{theo}
	If $V$ is a finite-dimensional vector space over an algebraically closed field $F$, and $L : V \to V$ is a linear map, then $L$ has at least one eigenvector.
	\label{eigenvectheo}
\end{theo}
\begin{proo*}
	TODO
\end{proo*}

These basic results on the trace of a matrix will be used extensively in our study of character theory.
\begin{prop}
\label{7}
	Let $A,B \in \M_n(F)$. Then
	\begin{mitem}
		\item $\Tr(A + B) = \Tr(A) + \Tr(B)$,
		\item $\Tr(AB) = \Tr(BA)$.
	\end{mitem}

	Also, if $T$ is invertible then $\Tr(TAT^{-1}) = \Tr(A)$.
\end{prop}
\begin{proo*}
TODO
\end{proo*}

Projections will be required for our proof of Maschke's Theorem.
\begin{defi}[Projection \cite{2}]
	Linear map $\pi$ from $V$ to a subspace $W$ is called a \textit{projection} if it satisfies $\pi^2 = \pi$, $\Im(\pi) = W$, $\pi:_W = \Id_W$.
\end{defi}
\begin{theo}
	Given a projection $\pi$ of $V$ onto a subspace $W$, we have $V = \Ker(\pi) \oplus \Im(\pi)$.
%	\label{projtheo}
\end{theo}
\begin{proo*}
	TODO
\end{proo*}
\begin{prop}
	Given a projection $\pi : V \to W$, we have $\Tr(\pi) = \dim(W)$. \label{Trproj}
\end{prop}
\begin{proo*}
TODO
\end{proo*}




Unitary maps will give rise to unitary representations, which we will examine when introducing representation theory.
\begin{defi}[Unitary map]
	A linear map $L : V \to W$ between inner pro spaces $V,W$ is said to be \textit{unitary} if $\langle v_1, v_2\rangle  = \langle L(v_1), L(v_2) \rangle\;\text{ for all } v_1,v_2 \in V$.
	We denote the unitary maps of a vector space $V$ as $\U(V)$, and for maps over an $n$-dimensional vector space over $\mathbb{C}$, we have $\U(V) \cong U_n(\mathbb{C}) \coloneqq \Brace{A \in \M_n(\mathbb{C}) \;:\; A^\dagger = A^{-1}}$ 
\end{defi}

\begin{exam}[\cite{1}]
	For the maps in $\GL_1(\mathbb{C})$, a complex number $z$ is unitary if $\bar{z} = z^1 \text{ implies } z\bar{z} = \abs{z}^2 = 1 \text{ implies } z \in \mathbb{T}$, where $\mathbb{T} = \Brace{z \in \mathbb{C}\;:\;\abs{z} = 1}$ is the unit circle. Then $\U_1(\mathbb{C}) = \mathbb{T}$
	\label{unitaryexam}
\end{exam}
\fi
\end{document}
