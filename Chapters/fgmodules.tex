\documentclass[../Project.tex]{subfiles}
\begin{document}
\newpage
\section{$FG$-Modules}
\subsection{Basic Definitions}
We will generalize the results so far in terms of $FG$-modules.
\begin{defi}[$FG$-module \cite{2}]
	For a vector space $V$ over a field $F$, and a group $G$, we say $V$ is an $FG$-module with respect to a multiplication operation $g \cdot v$ for $v \in V,\;g \in G$ if the following axioms are satisfied:
	\begin{enumerate}
		\item $g \cdot v \in V$,
		\item $(gg') \cdot v = g \cdot (g' \cdot v)$,
		\item $1 \cdot v = v$ (where $1$ is the identity element of $G$),
		\item $g \cdot (\alpha v) = \alpha(g \cdot v)$,
		\item $g \cdot (v + v') = g\cdot v + g \cdot v'$,
	\end{enumerate}
	$\forall$ $v,v' \in V,\;g,g' \in G$. As with group operations we will neglect the `$\cdot$' for ease of reading: $gv \coloneqq g \cdot v$.
\end{defi}

Note that axioms $(1)$,$(4)$,$(5)$ imply that $g \;\cdot : V \to G$ such that $v \mapsto gv$ is a linear endomorphism \cite{2}.

\begin{defi}[$FG$-module with chosen basis \cite{2}]
	Given an $FG$-module $V$ with a finite $n$ dimensional basis $\mathcal{B}$, we denote the matrix of the endomorphism $v \mapsto gv$ with respect to $\mathcal{B}$ as $\Brack{g}_\mathcal{B}$.\\
\end{defi}

\begin{theo}[\cite{2}]
	Let $V$ be an $n$ dimensional vector space over $F$, and $\phi : G \to \GL(V)$ be a group representation. $V$ becomes an $FG$-module by defining multiplication with $gv = \phi_g(v) \;\forall g\in G,\;v \in V$.\\

	We can see that given a basis $\mathcal{B}$ we have $\phi_g = \Brack{g}_\mathcal{B}$.
\end{theo}

\begin{proo*}[\cite{2}]
	Given an $n$ dimensional vector space $V$ over $F$, and a representation $\phi : G \to \GL(V)$ we have
	\begin{enumerate}
		\item $\phi_g(v) \in V$,
		\item $\phi_{gg'}(v) = \phi_g\phi_{g'}(v)$,
		\item $\phi_1{v} = v$ (since $\phi$ is a homomorphism it maps identity to identity),
		\item $\phi_g(\alpha v) = \alpha(\phi_g v)$,
		\item $\phi_g(v + v') = \phi_g(v) + \phi_g(v')$,
	\end{enumerate}
	$\,\forall v,v' \in V,\; \alpha \in F,\; g \in G$. Hence $gv \coloneqq \phi_g(v)$ allows $V \cong F^n$ becomes $FG$-module. $\blacksquare$
\end{proo*}



\begin{theo}[\cite{2}]
	Given and $FG$-module $V$ with basis $\mathcal{B}$, the function $g \mapsto \Brack{g}_\mathcal{B}$ is a representation of of $G$.
\end{theo}
\begin{proo*}[\cite{2}]
	Given an $FG$-module with basis $\mathcal{B}$. Since $(gg')v = g(g'v)\;\forall g,g' \in G,\;v \in V$, we have $[gg']_\mathcal{B} = [g]_\mathcal{B}[g']_\mathcal{B}$, then $\Brack{gg^{-1}}_\mathcal{B} = [1]_\mathcal{B} = [g]_\mathcal{B}[g']_\mathcal{B}$, and so $g \mapsto [g]_\mathcal{B}$ is a homomorphism from $G$ to $\GL_n(F)$.\;$\blacksquare$
\end{proo*}

\begin{defi}[Direct sum of $FG$-modules]
	\label{directsumoffg}
	Our direct sum of representations in Definition \ref{directsumofrep} extends to $FG$-modules, that is if we have $V = U \oplus W$ for $FG$-modules $V,U,W$ with chosen bases $\mathcal{B}_V,\mathcal{B}_W,\mathcal{B}_U$ respectively, then
	$$\Brack{g}_{\mathcal{B}_V} = \begin{pmatrix}\Brack{g}_{\mathcal{B}_U} & 0 \\ 0 & \Brack{g}_{\mathcal{B}_W}\end{pmatrix}.$$

More generally, given $FG$-modules $V,W_1,\dots,W_n$ with $V = W_1 \oplus \cdots \oplus W_n$, and bases $\mathcal{B}_V,{\mathcal{B}_{W_1}},\dots,{\mathcal{B}_{W_n}}$ respectively, we have
$$\begin{pmatrix} 
		[g]_{\mathcal{B}_V} = 
	    [g]_{\mathcal{B}_{W_1}}& 0 & \dots & 0 \\
	    0 & [g]_{\mathcal{B}_{W_2}} & \cdots & 0 \\
	    \vdots & \vdots& \ddots & \vdots\\
	    0 & 0  &\dots & [g]_{\mathcal{B}_{W_n}}
\end{pmatrix}.$$
\end{defi}



\begin{defi}[Trivial $FG$-module \cite{2}]
	The trivial $FG$-module is the 1-dimensional vector space $V$ over $F$ such that
	$gv = v\;\forall\,v \in V, g \in G$.
\end{defi}

\begin{defi}[Faithful $FG$-module \cite{2}]
	An $FG$-module $V$ is said to be faithful if and only if it is injective: $gv = v \implies g = 1\; \forall v \in V$.
\end{defi}


\begin{defi}[Regular $FG$-module \cite{2}]
	Let $G$ be a finite group of order $n$ and $F = \mathbb{C}$ or $\mathbb{R}$. The regular $FG$-module $V$ is the vector space over $F$ obtained using elements of $g$ as a basis, that is 
	$$V = \Brace{\sum_{i \in I}f_ig_i \;\vert\; f_i \in F,\,g_i \in G, I \subseteq \{1,\dots n\}},$$
	the set of finite sums of elements of $G$ with coefficients in $F$.
	It has the natural multiplication $vg = \sum\limits_{i \in I}f_ig_ig$.
	TODO verify this is a vector space

	Notice that the regular $FG$-module has dimension $\#G$.
\end{defi}

\begin{defi}[Regular representation \cite{2}]
	Let $G = \Brace{g_1 = 1_G,g_2, \dots, g_n}$ be a finite group of order $n$. The representation of the regular $FG$-module $V$ with basis $\mathcal{B} = G$ given by $g \mapsto \Brack{g}_\mathcal{B}$ is called the regular representation of $G$ over $F$.
\end{defi}

\begin{prop}[\cite{2}]
	The regular representation is faithful.
\end{prop}
\begin{proo*}
	Let $g \in G$, then $\forall v = \sum\limits_{i \in I}f_ig_i \in V$, suppose $vg = \sum\limits_{i \in I}f_ig_ig = \sum\limits_{i \in I}f_ig_i  = v$, then an identity basis term in a sum has $f1_Gg = f1_G$, so $g = 1_G$.
\end{proo*}

\begin{theo}[\cite{2}]
\label{5}
	Let $V$ be an $FG$-module with basis $\mathcal{B}$ and let $\phi$ be a representation $\phi : G \to \GL(V)$ such that $\phi_g = \Brack{g}_\mathcal{B}$.
	\begin{enumerate}
	\item If $\mathcal{B}'$ is another basis of $V$, and we have another representation $\psi : G \to \GL(V)$ such that $\psi_g = \Brack{g}_{\mathcal{B}'}$, then $\phi \sim \psi$.
	\item Conversely, if $\psi$ is a representation equivalent to $\phi$ then there is a basis $\mathcal{B}'$ of $V$ such that $\psi_g = \Brack{g}_{\mathcal{B}'}$.
	\end{enumerate}
\end{theo}

\begin{proo*}[\cite{2}]~ %%%
\vspace{-\topsep}
	\begin{enumerate}
		\item There exists a change of basis matrix $T$ such that $\Brack{g}_\mathcal{B} = T^{-1}\Brack{g}_{\mathcal{B}'}T$.
		\item Since $\phi \sim \psi$, $\exists T$ such that $\phi_g = T^{-1}\psi_gT\;\forall g \in G$. Let $\mathcal{B}'$ be a basis of $V$ such that the change of basis matrix from $\mathcal{B}$ to $\mathcal{B}'$ is $T$, then $\Brack{g}_\mathcal{B} = T^{-1}\Brack{g}_{\mathcal{B}'}T\;\forall g \in G$, and $\psi_g = \Brack{g}_{\mathcal{B}'}$.
	\end{enumerate}
	$\blacksquare$
\end{proo*}



We will now examine reducibility through the lense of $FG$-modules.

\begin{defi}[$FG$-submodule \cite{2}]
	Let $V$ be an $FG$-module. A subspace $W \leqslant V$ is an $FG$-submodule of $V$ is $gw \in W$ $\forall\,g \in G,\,w \in W$.
\end{defi}

\begin{defi}[Proper submodule \cite{2}]
	Every $FG$-module $V$ has at least two $FG$-submodules: $\Paren{0}$ and $V$. An $FG$-submodule $W < V$ is said to be proper if $W \neq \Paren{0},V$.
\end{defi}

\begin{defi}[\cite{2}]
A nonzero $FG$-module $V$ is irreducible if and only if it has no proper $FG$-submodules.
\end{defi}

We remark that the zero $FG$-module $V = \Paren{0}$ is regarded as neither reducible or irreducible, analogous to $1 \in \mathbb{N}$ being neither composite nor prime. (ASK)

\begin{defi}[$FG$-homomorphism \cite{2}]
	Let  $V$ and $W$ be $FG$-modules. An $FG$-homomorphism is a a linear function $\lambda : V \to W$ such that $\lambda(gv) = g\lambda(v)\;\forall g \in G, v \in V$. If $\lambda$ is a bijection then we say it is an $FG$-isomorphism, and $V \cong W$.
\end{defi}

\begin{prop}[\cite{2}]
\label{6}
Let $V,W$ be $FG$-modules. Then $V \cong W$ if and only if there exists a basis $\mathcal{B}_V$ of $V$ and $\mathcal{B}_W$ of $W$ such that
$$\Brack{g}_{\mathcal{B}_V} = \Brack{g}_{\mathcal{B}_W}.$$
\end{prop}
\begin{proo*}
TODO
\end{proo*}

\begin{defi}[$FG$-module projection \cite{2}]
	Given an $FG$-module $V$ and a collection of $FG$-submodules $\Brace{W_i}_{1 \leqslant i \leqslant n}$ such that $V = W_1 \oplus \cdots \oplus W_n$, for any vector $v = w_1 + \cdots + w_n$ where $v \in V$, $w_i \in W$ we define the projection $\pi_i : V \to W_i$ such that $\pi_i(v) = w_i$. This is a projection since $\pi_i^2(v) = \pi_i(w_i) = w_i$, $\Im(\pi_i) = W_i$, $\pi_i\mid_{W_i} = \Id_{W_i}$.
\end{defi}

\begin{prop}[\cite{2}]
\label{projishom}
	The above projection is an $FG$-module homorphism.
\end{prop}
\begin{proo*}
	$\pi_i : V \to W_i$ is linear since given $v,v' \in V$ and scalars $\alpha,\alpha' \in F$ we have $\pi_i(\alpha v + \alpha'v') = \pi(\alpha w_1 + \alpha'w_1' + \cdots + \alpha w_n + \alpha'w_n') = \alpha w_i + \alpha'w_i' = \alpha \pi_i(v) + \alpha'\pi_i(v')$ where $w_j,w_j' \in W_j$ $\forall 1 \leqslant j \leqslant n$. It's structure preserving since given $\pi_i(gv) = \pi_i(gw_1+ \cdots + gw_n) = gw_i = g\pi_i(v)$.
\end{proo*}

\begin{defi}[Image and kernel of $FG$-homomorphism]
	Like with other homomorphisms, given a $FG$-homomorphism $\phi : V \to W$
	$$\Im(\phi) = \Brace{\phi(v) \in W\;\vert\; v \in V},\quad \Ker(\phi) = \Brace{v \in V\;\vert\;\phi(v) = 0 \in W}.$$
\end{defi}

\begin{prop}
	For a $FG$-homomorphism $\phi : V \to W$, $\Im(\phi)$ is a $FG$-submodule of $W$ and $\Ker(\phi)$ is an $FG$-submodule of $V$.
\label{ImKerSub}
\end{prop}
\begin{proo*}
	Let $w = \phi(v) \in \Im(\phi)$. Then $\forall\,g \in G$, $gw = g\phi(v) = \phi(gv) \implies gw \in \Im(\phi)$. Let $v \in \Ker(\phi)$. Then $\forall\, g \in G$, $\phi(gv) = g\phi(v) = 0 \implies gv \in \Ker(\phi)$.
\end{proo*}



\newpage
\subsection{Maschke's Theorem for $FG$-Modules}
In the last section we covered Maschke's theorem in the case of a vector space over a field $F$ with $\Char F = 0$.
We use a more general case of Maschke's theorem in terms of $FG$-modules.


\begin{theo}[Maschke \cite{2}]
	Let $V$  be an $FG$-module where $G$ is a finite group, and $F$ a field of a characteristic such that $\Char(F) \nmid \#{G}$. If there exists an $FG$-submodule $W < V$ then there exists an $FG$-submodule $U$ such that $V = W \oplus U$. \label{3}
\end{theo}


\begin{defi}[Completely reducible $FG$-module \cite{2}]
	Let $V$ be an $FG$-module. $V$ is said to be completely reducible if and only if $V = W_1 \oplus \cdots \oplus W_k$ where $U_i$ is an irreducible $FG$-submodule of $V\;\forall\,1 \leqslant i \leqslant k$.
\end{defi}

\begin{prop}
	Maschke's theorem \ref{3} implies our earlier description Maschke's theorem \ref{4} where the field is $\mathbb{C}$.
\end{prop}

\begin{proo*}[\cite{2}]
	We have that $\Char\mathbb{C} = 0$. Let $V$ be an $n$-dimensional non-zero $FG$-module, with finite $G$, and $F = \mathbb{C}$. We proceed by induction on $\dim V$. Suppose $\dim V = 1$, then $V$ is trivially irreducible. Suppose $V$ is completely reducible up to $\dim V = k$. Then for $\dim V = k+1$, if $V$ is irreducible then the result holds, else $\exists W < V$ such that $W \neq \Brace{0},V$. By Maschke's theorem \ref{3} $\exists U < V$ with $U \neq \Brace{0},V$ such that $V = W \oplus U$. Since $\dim W, \dim U \leqslant k < \dim V$, both $W$ and $U$ are completely reducible by the inductive hypothesis, then
	$$V = W_1 \oplus \cdots \oplus W_{i_W} \oplus U_1 \oplus \cdots \oplus U_{i_U}$$
	where $i_W,i_U \in \mathbb{N}$ and $W_j,U_j$ are irreducible $\forall j$. $\blacksquare$
\end{proo*}

With these results out the way, we proove Maschke's theorem for $FG$-modules.
\begin{proo*}[Maschke \cite{3}]
	We have a finite group $G$ and an $FG$-module $V$ over field $F$ with characteristic such that $\Char(F)\;\nmid\;\#G$. Let $W$ be an $FG$-submodule of $V$. We define $\pi_W : V \to W$ be the projection onto $W$ as a vector space. Take $\tilde{\pi}_W : V \to W$ such that
	$$\tilde{\pi}_W(v) \coloneqq \frac{1}{\#G}\sum_{g \in G}g\pi_W(g^{-1}v).$$

	We will verify that this is an $FG$-homomorphism, even though $\pi_W$ on its own is just a projection of vector spaces. Linearity follows from linearity of $\pi_W$ with
	\begin{align*}
		&\frac{1}{\#G}\sum_{g \in G}g\pi_W(g^{-1}(v + v')) = \frac{1}{\#G}\sum_{g \in G}g\pi_W(g^{-1}v) + \frac{1}{\#G}\sum_{g \in G}g\pi_W(g^{-1}v'),\\
		&\frac{1}{\#G}\sum_{g \in G}g\pi_W(\alpha g^{-1}v) = \alpha\frac{1}{\#G}\sum_{g \in G}g\pi_W(g^{-1}v).
	\end{align*}
	 Now we demonstrate that $\tilde{\pi}_W(w) = w$. Clearly $g^{-1}w \in W$ since $W$ is a submodule, which implies $\pi(g^{-1}w) = g^{-1}w$, then
	$$\tilde{\pi}_W(w) = \frac{1}{\#G}\sum_{g \in G}g\pi_W(g^{-1}w) = \frac{1}{\#G}\sum_{g \in G}gg^{-1}w = \frac{1}{\#G}\sum_{g \in G}w = w.$$
	Also, since $\pi_W$ has its image in $W$, $\tilde{\pi}_W$ also has its image in $W$. Then $\tilde{\pi}_W$ is a projection.\\

	We verify structure presevation, given an element $h \in G$,
	$$h\tilde{\pi}_W(v) = \frac{1}{\#G}\sum_{g \in G}hg\pi_W(g^{-1}v) = \frac{1}{\#G}\sum_{g \in G}(hg)\pi_W((hg)^{-1}hv).$$
	Now let $g' = hg$. Then summing over all $g$ is the same as summing over all $g'$ so
	$$\frac{1}{\#G}\sum_{g \in G})(hg)\pi_W((hg)^{-1}hv) = \frac{1}{\#G}\sum_{g' \in G}g'\pi_W(g'^{-1}hv) = \tilde{\pi}_W(hv)$$
	and $\tilde{\pi}_W$ is structure preserving, and hence an $FG$-homomorphism.\\

	By Proposition \ref{ImKerSub}, $\Ker(\tilde{\pi}_W)$ is an $FG$-submodule. By Theorem \ref{projtheo} we have
	$$V = \Im(\tilde{\pi}_W) \oplus \Ker(\tilde{\pi}_W) = W \oplus \Ker(\tilde{\pi}_W). \;\blacksquare$$ 
	\end{proo*}
	
	Notice that the finite $G$ condition is required for $\dfrac{1}{\#G}$ to be defined, also if  $\Char(F) \mid \#G$ then $\dfrac{1}{\#G}$ is undefined since $\#G \equiv 0$ in $F$.\\

	While this proof requires this condition on $\#G$, a  natural question would be if Maschke's theorem will ever hold for $\Char(F) \mid \#G$. As it turns out the answer is no.

\begin{exam}
	We present an example of a group and representation such that $\Char(F) \mid \#{G}$ and Maschke's theorem does not hold. For the purpose of contradiction assume Maschke's theorem holds for $FG$-modules for fields of all characteristics.\\

	Let $C_3 = \Brace{1, g, g^2}$ be the cyclic group and $\overline{\mathbb{F}}_3 = \bigcup\limits_{n \in \mathbb{N}}\mathbb{F}_{3^n}$ be the algebraic closure of the finite field of three elements. Then $\Char(\overline{\mathbb{F}}_3) = 3\mid 3 = \# C_3$.
	We define the representation $\phi : C_3 \to \GL(V)$ with vector space $V$ over $\overline{\mathbb{F}}_3$ and basis $\mathcal{B} = \Brace{e_1,e_2,e_3}$ such that in matrix form we have the following linear maps
	$$\phi_1 = \begin{pmatrix}1 & 0 & 0\\0 & 1 & 0 \\0 & 0 & 1\end{pmatrix},\quad \phi_g = \begin{pmatrix}0 & 0 & 1\\1 & 0 & 0\\0 & 1 & 0\end{pmatrix},\quad \phi_{g^2} = \begin{pmatrix}0 & 1 & 0\\0 & 0 & 1\\1 & 0 & 0\end{pmatrix},$$
	or more specifically
	\begin{align*}
	&\phi_1(\alpha_1e_1 + \alpha_2e_2 + \alpha_3e_3) = (\alpha_1e_1+\alpha_2e_2 + \alpha_3e_3)\\
	&\phi_g(\alpha_1e_1+\alpha_2e_2+\alpha_3e_3) = (\alpha_3e_1+\alpha_1e_2+\alpha_2e_3),\\
	&\phi_{g^2}(\alpha_1e_1 + \alpha_2e_2 + \alpha_3e_3) = (\phi_g)^2(\alpha_1e_1,\alpha_2e_2,\alpha_3e_3) = (\alpha_2e_1 + \alpha_3e_2 + \alpha_1e_3)\\
	\forall\,\alpha_i \in \overline{\mathbb{F}}_3.
	\end{align*}

	Let $W = \overline{\mathbb{F}}_3\Brace{e_1+e_2+e_3}$. Then $\forall \alpha \in \overline{\mathbb{F}}_3$, $\phi_g(\alpha(e_1 + e_2 + e_3)) = \alpha\phi_g(e_1 + e_2 + e_3) = \alpha(e_1 + e_2 + e_3) \implies \phi_{g^2}(\alpha(e_1 + e_2 + e_3)) = \alpha(e_1 + e_2 + e_3)$. So $\forall w \in W$, $\phi_g(w) = w \implies \phi_{g^2}(w) = w$, and $W$ is a $C_3$-invariant subspace, hence by Maschke's theorem there exists another $C_3$-invariant subspace $U$ such that $V = W \oplus U$.\\

	By theorem \ref{eigenvectheo}, both $W$ and $U$ have at least one eigenvector for $\phi_g$, and hence $V$ must have at least two eigenvectors for $\phi_g$. Then calculating the eigenvectors of $\phi_g$ we have
	\begin{align*}
		\det{\begin{pmatrix}-\lambda & 0 & 1\\1 & -\lambda & 0\\0 & 1 & -\lambda \end{pmatrix}} = 0 \implies
		-(\lambda^3 + 1) = -(\lambda - 1)^3 = 0 \implies \lambda = 1.
	\end{align*}
	TODO finish with eigenvectors

\iffalse
	Then we find there is only one eigenvector corresponding to the single eigenvalue
	$$\begin{pmatrix}0 & 0 & 1\\1 & 0 & 0\\0 & 1 & 0\end{pmatrix}\begin{pmatrix}\alpha_1\\\alpha_2\\\alpha_3\end{pmatrix} = \begin{pmatrix}0\\0\\0\end{pmatrix} \implies \begin{pmatrix}\alpha_1\\\alpha_2\\\alpha_3\end{pmatrix} = \begin{pmatrix}0\\0\\0\end{pmatrix}.$$

	Hence Maschke's theorem does not hold.
	\fi
\end{exam}


\newpage
\subsection{Schur's Lemma}
\begin{lemm}[Schur \cite{2}]
	Let $V,W$ be irreducible $\mathbb{C}G$-modules.
	\begin{enumerate}
		\item For a $\mathbb{C}G$-homomorphism $\phi : V \to W$, either $\phi$ is a $\mathbb{C}G$-isomorphism or $\phi(v) = 0$ $\forall v \in V$.
		\item For a $\mathbb{C}G$-isomorphism $\phi : V \to V$, $\phi$ is a scalar multiple of the identity isomorphism $\phi = \alpha \Id_V$.
	\end{enumerate}
\end{lemm}

\begin{proo*}[\cite{2}]~ %%%
\vspace{-\topsep}
\begin{enumerate}
	\item Suppose $\exists \,v \in V$ such that $\phi(v) \neq 0$. Then $\Im(\phi) \neq \Brace{0}$. By proposition \ref{ImKerSub} we know $\Im(\phi)$ is a $\mathbb{C}G$-submodule of $W$, but $W$ is irreducible so $\Im(\phi) = W$ and $\phi$ is surjective. Also by proposition \ref{ImKerSub} $\Ker(\phi)$ is a $\mathbb{C}G$-submodule of $V$ and since $\Ker(\phi) \neq V$ and $V$ is irreducible, $\Ker(\phi) = \Brace{0}$ so $\phi$ is injective and hence a $\mathbb{C}G$-isomorphism.
	\item By theorem \ref{eigenvectheo} we have $\phi : V \to V$ must have at least one eigenvalue $\lambda \in \mathbb{C}$, then $\Ker(\phi - \lambda\Id_V) \neq \Brace{0}$. $\Ker(\phi - \lambda\Id_V)$ is a $\mathbb{C}G$-submodule of $V$, but $V$ is irreducible, so $\Ker(\phi - \lambda\Id_V) = V$ and $(\phi - \lambda\Id_V)v = 0\;\forall v \in V$, then $\phi = \lambda\Id_V$. $\blacksquare$
\end{enumerate}
\end{proo*}

\begin{prop}[\cite{2}]
	For a $\mathbb{C}G$-module $V$, if every $\mathbb{C}G$-endomorphism on $V$ is a scalar multiple of $\Id_V$ then $V$ is irreducible.
\end{prop}

\begin{proo*}[\cite{2}]
	Suppose for purpose of contradiction that $V$ is a reducible $\mathbb{C}G$-module where every $\mathbb{C}G$ endomorphism is a scalar multiple of the identity. Then there exists a proper $\mathbb{C}G$-submodule $W < V$, and by Maschke's theorem there exists a proper $FG$-submodule $U < V$ such that $V = U \oplus W$.

	Then by Proposition \ref{projishom}, $\pi_W : V \to V$ such that $\pi(u + w) = w$ $\forall w \in W,\,u \in U$ is a $\mathbb{C}G$-homomorphism. But $\pi_W$ isn't a scalar multiple of $0$.\\

	Then by contradiciton $V$ is irreducible. $\blacksquare$
\end{proo*}

\begin{theo}[Fundamental theorem of finite abelian groups \cite{4}]
	Let $G$ be a finite abelian group. Then $G$ is isomorphic to a direct product of cyclic groups $C_{p_1}^{n_1} \oplus \cdots C_{p_m}^{n_m}$ each of which has an order equal to a prime power.
\end{theo}

\begin{proo*}
	TODO
\end{proo*}

\begin{prop}[\cite{2}]
	If $G$ is finite abelian then every irreducible $\mathbb{C}G$-module $V$ has dimension $1$.
\end{prop}
\begin{proo*}
	Let $G$ be a finite abelian and $V$ and irreducible $\mathbb{C}G$-module with basis $\mathcal{B}$. $\forall g,g' \in G,\,v \in V,\alpha,\alpha' \in \mathbb{C}$ we have
	$$gg' = g'g \implies g'(gv) = (gg'v),\quad g(\alpha v + \alpha'v') = \alpha(gv) + \alpha'(gv').$$
	So the map $\phi_g : V \to V$ such that $v \mapsto gv \;\forall v \in V$ is a $\mathbb{C}G$-endomorphism for any $g \in G$. Then by Schur's lemma (1) either
	\begin{enumerate}
		\item $gv = 0 = 0v\, \forall v \in V$,
		\item or $\phi_g$ is a $\mathbb{C}G$-automorphism and $gv = \lambda v$ for some $\lambda \in \mathbb{C}$.
	\end{enumerate}
	Then  $\exists \lambda_g \in \mathbb{C}$ for a given $g$ such that $gv = \lambda_g v$ $\forall v \in G$. Every subvectorspace $W < V$ is also an $\mathbb{C}G$-submodule since $gw = \lambda_gw \in W\; \forall w \in W$. Then $\dim(V) = 1$, else we could choose any basis element $e_i \in \mathcal{B}$ an get a $\mathbb{C}G$-submodule $\mathbb{C}\Brace{e_i} \neq V$. $\blacksquare$
\end{proo*}

\end{document}
